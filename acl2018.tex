%
% File acl2018.tex
%
%% Based on the style files for ACL-2017, with some changes, which were, in turn,
%% Based on the style files for ACL-2015, with some improvements
%%  taken from the NAACL-2016 style
%% Based on the style files for ACL-2014, which were, in turn,
%% based on ACL-2013, ACL-2012, ACL-2011, ACL-2010, ACL-IJCNLP-2009,
%% EACL-2009, IJCNLP-2008...
%% Based on the style files for EACL 2006 by
%%e.agirre@ehu.es or Sergi.Balari@uab.es
%% and that of ACL 08 by Joakim Nivre and Noah Smith

\documentclass[11pt,a4paper,onecolumn]{article}
\usepackage[hyperref]{acl2018}
\usepackage{times}
\usepackage{latexsym}

\usepackage{url}

\aclfinalcopy % Uncomment this line for the final submission

%\setlength\titlebox{5cm}
% You can expand the titlebox if you need extra space
% to show all the authors. Please do not make the titlebox
% smaller than 5cm (the original size); we will check this
% in the camera-ready version and ask you to change it back.

\newcommand\BibTeX{B{\sc ib}\TeX}

\title{Automatic Summarization: State-of-the-art review}

\author{First Author \\
  Affiliation / Address line 1 \\
  Affiliation / Address line 2 \\
  Affiliation / Address line 3 \\
  {\tt email@domain} \\\And
  Second Author \\
  Affiliation / Address line 1 \\
  Affiliation / Address line 2 \\
  Affiliation / Address line 3 \\
  {\tt email@domain} \\}

\date{}

\begin{document}
\maketitle
\begin{abstract}
This paper provides an overview of the most prominent algorithms for automatic summarization.
\end{abstract}

\section{Introduction}

With large amounts of text available online, the amount of which is growing rapidly everyday, summarizing this content becomes necessary to help users handle such information overload and better perceive information.
Automatic summarization is the process of automatically shortening the content of an information source in a way that retains its most important information.
The goal of summarizers is to produce concise and fluent summary which would allow people to understand the content of the input without reading the entire text.
A good summary must be concise and fluent, capture all the important topics, but not contain repetitions of the same information.
For a long time extractive techniques of text summarization have been the primary focus of research in the field.
However, in the past years there have been less major advances in extractive text summarization.
The alternative approach of abstractive summarization...

\section{Extractive methods}
Extractive summarization methods produce summaries by concatenating several sentences (text units) from the text being summarized exactly as they occur.
The main task of such systems is to determine which sentences are important and should therefore be included in the summary.
For many years extractive methods have been the main focus of researchers in the text summarization community.

In the past years there has not been any substantial advancements in extractive text summarization, most publications only proposed some small improvements to the already long-existing extractive methods.
Researchers [] assume that extractive summarization methods may have achieved their peak and propose two possible research advancement options:  1) making ensembles of extractive methods and 2) focusing on abstractive techniques.

\section{Abstractive methods}
When people produce summaries, in order to make them optimal in terms of content and linguistic quality, they tend to edit the text rather than just copy sentences from it as they are.
In order to emulate this process, generation of abstractive summaries is necessary.
Abstractive summarization methods produce summaries by rewriting the content in an input, as opposed to extractive methods that simply extract and concatenate important text units from it.
Abstractive summarization requires a deeper semantic and discourse interpretation of the text, as well as a novel text generation process.

\section{Evaluation methods}


% include your own bib file like this:
%\bibliographystyle{acl}
%\bibliography{acl2018}
\bibliography{acl2018}
\bibliographystyle{acl_natbib}

\appendix

\end{document}
